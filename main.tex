% -- Encoding UTF-8 without BOM
% -- XeLaTeX => PDF (BIBER)

\documentclass[]{cv-style}          % Add 'print' as an option into the square bracket to remove colours from this template for printing.
                                    % Add 'espanol' as an option into the square bracket to change the date format of the Last Updated Text

\sethyphenation[variant=british]{english}{french} % Add words between the {} to avoid them to be cut

\begin{document}

\header{George}{Abitbol}           % Indiquez votre nom pour l'en-tête
% \lastupdated

%----------------------------------------------------------------------------------------
%	SIDEBAR SECTION  -- aside est la colonne à gauche pour les informations
%----------------------------------------------------------------------------------------

\begin{aside}
% --
\section{Contact}
255 Route de Pétaouchnock
Pétaouchnock
~
France
~
01.02.03.04.05
~
% -- href permet d'intégrer un lien , ici le mailto permet au recruteur
% -- d'envoyer un courriel directement
 \href{mailto:george.abitbol@gmail.com}{george.abotbol@gmail.com}
~
~
% -- Nous définissons ici les sections
\section{Langues}
{\color{blue} $\varheartsuit$}
Francais
English-Espagnol
~
%
\section{Mobilité}
{\color{red} $\varheartsuit$}
Transport en commun
Train
Permis B en cours
%
\section{Caractère}
{\color{blue} $\varheartsuit$}
Patient
Volontaire
Ouvert
Disponible
Dynamique
Autonome,
Esprit d'initiatives
%
\section{Transverse}
{\color{orange} $\varheartsuit$}
Outils bureautique
Traitement de texte
Tableur
Présentation
~
\section{Loisirs}
{\color{green} $\varheartsuit$}
Rando/Balade
Lecture
Free Fight
Pétanque
~
%
\end{aside}
%----------------------------------------------------------------------------------------
%	LES APTITUDES
%----------------------------------------------------------------------------------------

\section{La  Classe Américaine}

%----------------------------------------------
	{Recueillir les informations, s’informer sur les éléments du contexte}\\
	{Mettre en œuvre les conditions favorables à l’activité}\\
	{Mettre en œuvre des activités d’éveil }\\
    {Réaliser des soins du quotidien-accompagner l’enfant dans ses apprentissages}\\
	{Appliquer des protocoles liés à la santé de l’enfant }\\
	{Coopérer avec l’ensemble des acteurs-- Rendre compte (oral/écrit) d’une activité}
%----------------------------------------------------------------------------------------
%	EXP2RIENCES ET FORMATIONS
%----------------------------------------------------------------------------------------

\section{Formation et expériences}
\begin{entrylist}
\entry
{2014-2017}
{Bac Pro Cosmonaute Toulouse}
{Lycée de l'Espace Toulouse}
{\jobtitle{Formation en alternance de 3 ans}\\
	\begin{itemize}
		\item accompagner des publics non autonomes et \\développer une communication adaptée
		\item techniques de soins de confort
		\item règles d'hygiène et de sécurité
		\item intégrer les enjeux de nos territoires montagnards, \\péri-urbains et touristiques
		\item analyser des pratiques professionnelles françaises et étrangères
\end{itemize}}\\
\entry
{Juin 2017}
{CAP Tricot}
{Lycée de l'haleine Mouton}
{\jobtitle{Obtenu en candidat libre}\\
	\begin{itemize}
		\item Examen écrit sur la PSE \\(Protection Santé et Environnement
		\item Oral sur la réalisation du stage (activité réalisées)
		\item Appliquer des protocoles liés à la santé de l’enfant
		\item Mettre en œuvre des activités d’éveil
		\item accompagner l’enfant dans ses apprentissages
\end{itemize}}\\
\entry
{Juin 2017}
{Cobaye}
{NASA Cap Carnaval}
{\jobtitle{Stagiaire}\\
	\begin{itemize}
		\item Service du repas des enfants
		\item Nettoyage à la fin du service
		\item Animation et TPE (Temps Péri-Educatifs)
	\end{itemize}}\\
	%------------------------------------------------
\entry
{Mai 2017}
{Cobaye}
{ESA Toulouse}
{\jobtitle{Remplacement }\\
	\begin{itemize}
		\item Encadrement/Service du repas des enfants et surveillance
		\item Surveillance en cours de récréation et activitées diverses
		\item soins du quotidien-accompagner l’enfant dans ses apprentissages
	\end{itemize}}\\
%------------------------------------------------
\entry
{2016--2017}
{Apnéiste}
{Piscine Municipale de Toulouse}
{\jobtitle{Testeur de bouteille d'oxygène}\\
	\begin{itemize}
		\item Accueil, Prise en charge, accompagnement, soins
		\item Jeu d’éveil, Jeu pédagogique; sociabilisation
		\item Soins du quotidien-accompagner l’enfant dans ses apprentissages
\end{itemize}}\\
%-----------------------------------------------
\end{entrylist}
%-----------------------------------------------
% -- Lorsque le CV déborde sur une 2° pages il faut indiquer newpage
% -- j'intègre à nouveau le header (en-tête) ainsi que le aside
% -- de cette manière le recruteur à toujours votre nom et vos coordonnées
\newpage
\header{George Abitbol}
%------------------------------------------------

%-----------------------------------------------
\begin{aside}
\section{Contact}
255 Route de Pétaouchnock
France
~
01.02.03.04.05
~
george.abitbol
@gmail.com
\end{aside}
%------------------------------------------------
\begin{entrylist}
%------------------------------------------------

\entry
  {10/2016--12/2016}
  {Catcheur}
  {Cirque d'Hiver et variété}
  {\jobtitle{Stagiaire}\\
  \begin{itemize}
  	\item Accueil, Prise en charge, accompagnement, soins
  	\item Jeu d’éveil, Jeu pédagogique ;sociabilisation
  	\item Soins du quotidien-accompagner l’enfant dans ses apprentissages
  \end{itemize}}\\
%-----------------------------------------------

\entry
  {09/2015--12/2015}
  {Karateka}
  {Dojo de la main plate}
  {\jobtitle{Stagiaire}\\
  \begin{itemize}
    \item Accueil, Prise en charge, accompagnement, soins
    \item Jeu d’éveil, Jeu pédagogique ;sociabilisation
    \item Soins du quotidien-accompagner l’enfant dans ses apprentissages
  \end{itemize}}
%------------------------------------------------

\end{entrylist}

%----------------------------------------------------------------------------------------
%	CENTRE D'INTERET
%----------------------------------------------------------------------------------------

\section{Intérêts}
  \vspace{-0.1cm}

\textbf{Professionel:} Réaliser des soins du quotidien-accompagner l’enfant dans ses apprentissages.Mettre en œuvre les conditions favorables à l’activité;
Mettre en œuvre des activités d’éveil;
Mettre en œuvre les conditions favorables à l’activité\\
\textbf{personnel:} Randonnée, balade, lecture
%----------------------------------------------------------------------------------------

\end{document}
